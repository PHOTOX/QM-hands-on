Up to this point, we have taken the time-independent perspective on quantum mechanics focusing on methods for calculating stationary states. However, the world around us is in constant motion and for its full description, we need to have a time-dependent perspective. In this chapter, we shall take a look at time evolution in quantum mechanics and show how to describe dynamics with numerical methods. We will focus only on one-dimensional systems that bring understanding of phenomena and are easy to implement.

\section{Theoretical background}

The time evolution in quantum mechanics is governed by the \acrfull{tdse},
\begin{equation}
    i\hbar\frac{\dd\psi(x,t)}{\dd t} = \hat{H}(x)\psi(x,t) \, ,
    \label{eq:tdse}
\end{equation}
where $i$ is the imaginary unit, $\hbar$ is the reduced Planck constant, $\psi$ is the wave function, $\hat{H}$ is the Hamiltonian, $t$ is time and $x$ is the position of the particle. The solution of the \acrshort{tdse} can be in general written as
\begin{equation}
    \psi(x,t) = \hat{U}(t)\psi(x,0) \, .
    \label{eq:U}
\end{equation}
The operator $\hat{U}$ is called \textit{propagator} or the \textit{time evolution operator}. Application of the evolution operator to a wave function is equivalent to evolving the wave function in time. The knowledge of the evolution operator is, therefore, elementary for a description of dynamics in quantum systems.

If the Hamiltonian $\hat{H}$ is time-independent, meaning that $\hat{H}$ is not a function of time, we can formally find a solution of equation~\eqref{eq:tdse} in the following form:
\begin{equation}
    \psi(x,t) = \e^{-\frac{i}{\hbar}\hat{H}(x)t} \psi(x,0)\, .
    \label{eq:tdse2}
\end{equation}
Note that the result resembles the separation of variables for differential equations, only for equations with operators. Comparing equations~\eqref{eq:tdse2} and \eqref{eq:U}, we see that the evolution operator 
Considering the Hamiltonian to be time-independent works for all systems that are not perturbed externally, such as by electromagnetic radiation or scattering atoms passing by. 

\textbf{All these claims to be verified!!!}


\section{Numerical solution: the split-operator method}

propagators:

Fourier transform necessary
\begin{lstlisting}[language=Python, style=mystyle2]
p = 2*np.pi*np.fft.fftfreq(ngrid, d=dx)
\end{lstlisting}

\subsection{Exercise}

Following is the incomplete code that students are supposed to complete.

\lstset{style=mystyle}
\lstinputlisting[caption=Exercise: quantum dynamics in real-time,language=Python]{codes/exercises/rt_quantum_dynamics.py}




\clearpage
\textbf{THIS IS FOR NEXT CHAPTERS}
\section{Correlation function}



\section{Applications}

\begin{itemize}
    \item coherent and squeezed gaussian states
    \item optimization of stationary states
    \item absorption spectrum
\end{itemize}



