\chapter{Hartree--Fock Method}

The \acrshort{hf} method is a foundational approach in quantum chemistry, aimed at solving the electronic structure problem in molecules by determining an optimal wavefunction. This method simplifies the complex interactions of electrons through a mean-field approximation, where each electron moves in an average field created by all others. This allows for the use of a single set of orbitals, leading to the construction of the Fock operator and iterative solutions to one-electron equations.

However, \acrshort{hf} has notable limitations. Its reliance on a single-determinant wavefunction means it struggles to account for electron correlation. This leads to inaccuracies in energy predictions, particularly for systems with strong electron interactions, such as transition metal complexes or molecules with delocalized electrons.

\section{Theoretical Background}

Our primary objective is to solve the Schrödinger equation in the form

\begin{equation}
\hat{\mathbf{H}}\ket{\Psi}=E\ket{\Psi}
\end{equation}

where \(\hat{\mathbf{H}}\) denotes the molecular Hamiltonian operator, \(\ket{\Psi}\) represents the molecular wavefunction, and \(E\) is the total energy of the system. The \acrshort{hf} approximates the total wavefunction \(\ket{\Psi}\) as a single Slater determinant, expressed as

\begin{equation}
\ket{\Psi}=\ket{\chi_1\chi_2\cdots\chi_N}
\end{equation}

where \(\chi_i\) denotes a spin orbital, and \(N\) is the total number of electrons. The goal of the \acrshort{hf} method is to optimize these orbitals in order to minimize the system's total energy, thereby providing a reliable estimate of the electronic structure.

In the \acrfull{rhf} method, we impose a constraint on electron spin, which allows us to work with spatial orbitals instead of spin orbitals. This reformulation expresses the Slater determinant in terms of spatial orbitals as

\begin{equation}
\ket{\Psi}=\ket{\Phi_1\Phi_2\cdots\Phi_{N/2}}
\end{equation}

where \(\Phi_i\) represents a spatial orbital. Notably, the \acrshort{rhf} method requires an even number of electrons to satisfy spin-pairing. In practice, atomic orbitals (whether spin or spatial) are typically expanded in a set of basis functions \(\left\lbrace\phi_i\right\rbrace\), which are often Gaussian functions, allowing for convenient computation with expansion coefficients. After performing some algebraic manipulations, the \acrshort{hf} method leads to the Roothaan equations, which are expressed as

\begin{equation}\label{eq:roothaan}
\mathbf{FC}=\mathbf{SC}\bm{\varepsilon}
\end{equation}

where \(\mathbf{F}\) is the Fock matrix, \(\mathbf{C}\) is the matrix of orbital coefficients, \(\mathbf{S}\) is the overlap matrix, and \(\bm{\varepsilon}\) represents the orbital energies. These matrices will be defined in detail later.

\section{Implementation of the Restricted Hartree--Fock Method}

To begin, we define the core Hamiltonian, also known as the one-electron Hamiltonian. This component of the full Hamiltonian excludes electron-electron repulsion and is expressed in index notation as

\begin{equation}\label{eq:hamiltonian}
H_{\mu\nu}^{\mathrm{core}}=T_{\mu\nu}+V_{\mu\nu}
\end{equation}

where \(\mu\) and \(\nu\) are indices of the basis functions. Here, \(T_{\mu\nu}\) represents a kinetic energy matrix element, while \(V_{\mu\nu}\) denotes a potential energy matrix element. These matrix elements are defined as

\begin{align}
T_{\mu\nu}&=\braket{\phi_{\mu}|\hat{T}|\phi_{\nu}} \\
V_{\mu\nu}&=\braket{\phi_{\mu}|\hat{V}|\phi_{\nu}}
\end{align}

Additionally, the overlap integrals, which describe the extent of overlap between basis functions, are given by

\begin{equation}\label{eq:overlap}
S_{\mu\nu}=\braket{\phi_{\mu}|\phi_{\nu}}
\end{equation}

and another essential component are the two-electron repulsion integrals, defined as

\begin{equation}\label{eq:coulomb}
J_{\mu\nu\kappa\lambda}=\braket{\phi_{\mu}\phi_{\mu}|\hat{J}|\phi_{\kappa}\phi_{\lambda}}
\end{equation}

which play crucial roles in the \acrshort{hf} calculation. All of these integrals over (Gausssian) basis functions are usually calculated using analytical expressions.\cite{10.1016/S0065-3276!08!60019-2} The solution to the Roothaan equations \eqref{eq:roothaan} requires an iterative procedure, since the Fock matrix defined as

\begin{equation}\label{eq:fock}
F_{\mu\nu}=H_{\mu\nu}^{\mathrm{core}}+D_{\kappa\lambda}\left(J_{\mu\nu\kappa\lambda}-\frac{1}{2}J_{\mu\lambda\kappa\nu}\right)
\end{equation}

depends on the unknown density matrix \(\mathbf{D}\), defined as

\begin{equation}\label{eq:hf_density}
D_{\mu\nu}=2C_{\mu i}C_{\nu i}
\end{equation}

This iterative procedure is executed through the \acrfull{scf} method. An initial guess is made for the density matrix \(\mathbf{D}\) (usually zero), the Roothaan equations \eqref{eq:roothaan} are solved and the density matix is updated using the equation \eqref{eq:hf_density}. The total energy of the system is then calculated using the core Hamiltonian and the Fock matrix as

\begin{equation}
E=\frac{1}{2}D_{\mu\nu}(H_{\mu\nu}^{\mathrm{core}}+F_{\mu\nu})+E_{\mathrm{nuc}}
\end{equation}

After convergence of both the density matrix and total energy, the process concludes, yielding the optimized molecular orbitals. The total energy of the system also includes the nuclear repulsion energy, which is given by

\begin{equation}
E_{\mathrm{nuc}}=\sum_{A}\sum_{B<A}\frac{Z_{A}Z_{B}}{R_{AB}}
\end{equation}

where \(Z_A\) is the nuclear charge of atom \(A\), and \(R_{AB}\) is the distance between atoms \(A\) and \(B\).

\subsection{Direct Inversion in the Iterative Subspace}

In the \acrshort{hf} method, convergence of the density matrix and energy can be significantly accelerated by employing the \acrfull{diis} technique. \acrshort{diis} achieves this by storing Fock matrices from previous iterations and constructing an optimized linear combination that minimizes the current iteration's error. This approach is especially valuable in \acrshort{hf} calculations for large systems, where convergence issues are more common and challenging to resolve.

We start by defining the error vector of \(i\)-th iteration \(\mathbf{e}_i\) as

\begin{equation}
\mathbf{e}_i=\mathbf{S}_i\mathbf{D}_i\mathbf{F}_i-\mathbf{F}_i\mathbf{D}_i\mathbf{S}_i
\end{equation}

Our goal is to transform the Fock matrix \(\mathbf{F}_i\) as

\begin{equation}\label{eq:fock_extrapolate}
\mathbf{F}_i=\sum_{j=i-(L+1)}^{i-1}c_j\mathbf{F}_j
\end{equation}

where \(c_j\) are the coefficients that minimize the error matrix \(\mathbf{e}_i\) and \(L\) is the number of Fock matrices we store called the subspace size. To calculate the coefficients \(c_j\), we solve the set linear equations

\begin{equation}
\begin{bmatrix}
\mathbf{e}_1\cdot\mathbf{e}_1 & \dots & \mathbf{e}_1\cdot\mathbf{e}_{L+1} & 1 \\
\vdots & \ddots & \vdots & \vdots \\
\mathbf{e}_{L+1}\cdot\mathbf{e}_1 & \dots & \mathbf{e}_{L+1}\cdot\mathbf{e}_{L+1} & 1 \\
1 & \dots & 1 & 0 \\
\end{bmatrix}
\begin{bmatrix}
c_1 \\
\vdots \\
c_{L+1} \\
\lambda \\
\end{bmatrix}
=
\begin{bmatrix}
0 \\
\vdots \\
0 \\
1 \\
\end{bmatrix}
\end{equation}

After solving the linear equations, we use the coefficients \(c_j\) to construct the new Fock matrix \(\mathbf{F}_i\) according to the equation \eqref{eq:fock_extrapolate} and proceed as usual with the \acrshort{hf} calculation.

\subsection{Gradient of the Restricted Hartree--Fock Method}

If we perform the calculation as described above and get the density matrix \(\mathbf{D}\) we can evaluate the nuclear energy gradient as\cite{10.1002/9780470749593.hrs006}

\begin{equation}\label{eq:hf_gradient}
\end{equation}

where \(A\) is the index of an atom, \(i\) is the index of the coordinate, \(\left\lbrace\phi_A\right\rbrace\) is the set of all basis functions located at atom \(A\) and \(\mathbf{W}\) is energy weighed density matrix defined as

\begin{equation}
W_{\mu\nu}=2C_{\mu i}C_{\nu i}\varepsilon_i
\end{equation}

Keep in mind that the indices \(\kappa\) and \(\lambda\) in the gradient equation \eqref{eq:hf_gradient} are summed over all basis functions.

\section{\texorpdfstring{Integral Transforms to the Basis of Molecular Spinorbitals\label{sec:integral_transform}}{Integral Transforms to the Basis of Molecular Spinorbitals}}

To carry out most \acrfull{post-hf} calculations, it is essential to transform the integrals into the \acrfull{ms} basis. We will outline this transformation process here and refer to it in subsequent sections on \acrshort{post-hf} methods. The \acrshort{post-hf} methods in this document will be presented using the integrals in the \acrshort{ms} basis (and in its antisymmetrized form for the two-electron integrals) as this approach is more general.

All the integrals defined in the equations \eqref{eq:hamiltonian}, \eqref{eq:overlap}, and \eqref{eq:coulomb}, as well as Fock matrix in the equation \eqref{eq:fock} are defined in the basis of atomic orbitals. To transform these integrals to the \acrshort{ms} basis, we utilize the coefficient matrix \(\mathbf{C}\) obtained from the solution of the Roothaan equations \eqref{eq:roothaan}. This coefficient matrix \(\mathbf{C}\) is initially calculated in the spatial molecular orbital basis (in the \acrshort{rhf} calculation).

The first step involves expanding the coefficient matrix \(\mathbf{C}\) to the \acrshort{ms} basis. This transformation can be mathematically expressed using the tiling matrix \(\mathbf{P}_{n\times 2n}\), defined as

\begin{equation}
\mathbf{P}=
\begin{pmatrix}
e_1&e_1&e_2&e_2&\dots&e_n&e_n
\end{pmatrix}
\end{equation}

where \(e_i\) represents the \(i\)-th column of the identity matrix \(\mathbf{I}_n\). Additionally, we define the matrices \(\mathbf{M}_{n\times 2n}\) and \(\mathbf{N}_{n\times 2n}\) with elements given by

\begin{equation}
M_{ij}=1-j\bmod 2,N_{ij}=j \bmod 2
\end{equation}

The coefficient matrix \(\mathbf{C}\) in the \acrshort{ms} basis can be then expressed as

\begin{equation}
\mathbf{C}^{\mathrm{MS}}=
\begin{pmatrix}
\mathbf{CP} \\
\mathbf{CP}
\end{pmatrix}
\odot
\begin{pmatrix}
\mathbf{M} \\
\mathbf{N}
\end{pmatrix}
\end{equation}

where \(\odot\) denotes the Hadamard product. This transformed matrix \(\mathbf{C}^{\mathrm{MS}}\) is subsequently used to transform the two-electron integrals \(\mathbf{J}\) to the \acrshort{ms} basis as

\begin{equation}
J_{pqrs}^{\mathrm{MS}}=C_{\mu p}^{\mathrm{MS}}C_{\nu q}^{\mathrm{MS}}(\mathbf{I}_{2}\otimes_K(\mathbf{I}_{2}\otimes_K\mathbf{J})^{(4,3,2,1)})_{\mu\nu\kappa\lambda}C_{\kappa r}^{\mathrm{MS}}C_{\lambda s}^{\mathrm{MS}}
\end{equation}

where the superscript \((4,3,2,1)\) denotes the axes transposition and \(\otimes_K\) is the Kronecker product. This notation accommodates the spin modifications and ensures adherence to quantum mechanical principles. We also define the antisymmetrized two-electron integrals in physicists' notation as

\begin{equation}
\braket{pq||rs}=(J_{pqrs}^{\mathrm{MS}}-J_{psrq}^{\mathrm{MS}})^{(1,3,2,4)}
\end{equation}

For the transformation of the one-electron integrals such as the core Hamiltonian, the overlap matrix and also the Fock matrix, we use the formula

\begin{equation}
A_{pq}^{\mathrm{MS}}=C_{\mu p}^{\mathrm{MS}}(\mathbf{I}_{2}\otimes_K\mathbf{A})_{\mu\nu}C_{\nu q}^{\mathrm{MS}}
\end{equation}

where \(\mathbf{A}\) is an arbitrary matrix of one-electron integrals. Since many \acrshort{post-hf} methods rely on differences of orbital energies in the denominator, we define the tensors

\begin{align}
\varepsilon^{a}_{i}&=\varepsilon_i-\varepsilon_a \\
\varepsilon^{ab}_{ij}&=\varepsilon_i+\varepsilon_j-\varepsilon_a-\varepsilon_b \\
\varepsilon^{abc}_{ijk}&=\varepsilon_i+\varepsilon_j+\varepsilon_k-\varepsilon_a-\varepsilon_b-\varepsilon_c
\end{align}

for convenience. These tensors enhance code readability and efficiency, making it easier to understand and work with the underlying mathematical framework. Here and also throughout the rest of the document, the indices \(i\), \(j\) and \(k\) run over occupied orbitals, whereas the indices \(a\), \(b\) and \(c\) run over virtual orbitals.

\section{\texorpdfstring{Hartree--Fock Method and Integral Transform Coding Exercise\label{sec:hf_int_code_exercise}}{Hartree--Fock Method and Integral Transform Coding Exercise}}

This section provides Python code snippets for implementing the \acrshort{hf} method and transforming integrals to the \acrshort{ms} basis. The code uses the NumPy library for efficient numerical computations, and the exercises are designed to build familiarity with the \acrshort{hf} method and the \acrshort{ms} integral transformations.

Each exercise includes placeholders for you to fill. It is assumed that the variables \texttt{\lstinline!atoms!}, \texttt{\lstinline!coords!}, \texttt{\lstinline!S!}, \texttt{\lstinline!H!}, and \texttt{\lstinline!J!} are defined before the exercises. These variables represent the atomic numbers, atomic coordinates, overlap matrix, core Hamiltonian, and two-electron integrals, respectively. These variables can typically be obtained from the output of a quantum chemistry software package. If you would like to focus solely on coding you can save the \textattachfile[color=0 0 1]{molecule.xyz}{molecule file}, \textattachfile[color=0 0 1]{S_AO.mat}{overlap integrals}, \textattachfile[color=0 0 1]{H_AO.mat}{core Hamiltonian}, and \textattachfile[color=0 0 1]{J_AO.mat}{two-electron integrals} in the STO-3G basis to the same directory as the exercise code and load the variables using the Listing \ref{code:load_exercise} below. The \texttt{\lstinline!ATOM!} variable is a dictionary that maps the atomic symbols to atomic numbers.

\raggedbottom\begin{lstlisting}[language=Python, caption={Example loading of molecule and integrals over atomic basis functions into variables used throughout exercises. The snippet expects the molecule and integral files to b present in the same directory as the script.}, label=code:load_exercise]
# get the atomic numbers and coordinates of all atoms
atoms = np.array([ATOM[line.split()[0]] for line in open("molecule.xyz").readlines()[2:]], dtype=int)
coords = np.array([line.split()[1:] for line in open("molecule.xyz").readlines()[2:]], dtype=float)

# convert to bohrs
coords *= 1.8897261254578281

# load the integrals from the files
H, S = np.loadtxt("H_AO.mat", skiprows=1), np.loadtxt("S_AO.mat", skiprows=1); J = np.loadtxt("J_AO.mat", skiprows=1).reshape(4 * [S.shape[1]])
\end{lstlisting}

With all the variables defined, you can proceed to the \acrshort{hf} and integral transform exercises in the Listings \ref{code:hf_exercise} and \ref{code:int_exercise} below.

\raggedbottom\begin{lstlisting}[language=Python, caption={\acrshort{hf} method exercise code. The important variables like number of occupied orbitals, convergence threshold and matrix containers are already defined. The student is expected to fill the \acrshort{scf} loop and calculate nuclear repulsion energy from the atomic numbers and coordinates. The total energy is then automatically printed.}, label=code:hf_exercise]
"""
Here are defined some of the necessary variables. The variable "E_HF" stores the Hartree-Fock energy, while "E_HF_P" keeps track of the previous iteration's energy to monitor convergence. The "thresh" defines the convergence criteria for the calculation. The variables "nocc" and "nbf" represent the number of occupied orbitals and the number of basis functions, respectively. Initially, "E_HF" is set to zero and "E_HF_P" to one to trigger the start of the Self-Consistent Field (SCF) loop. Although you can rename these variables, it is important to note that certain sections of the code are tailored to these specific names.
"""
E_HF, E_HF_P, nocc, nbf, thresh = 0, 1, sum(atoms) // 2, S.shape[0], 1e-8

"""
These lines set up key components for our HF calculations. We initialize the density matrix as a zero matrix, and the coefficients start as an empty array. Although the coefficient matrix is computed within the while loop, it's defined outside to allow for its use in subsequent calculations, such as the MP energy computation. Similarly, the exchange tensor is accurately calculated here by transposing the Coulomb tensor. The "eps" vector, which contains the orbital energies, is also defined at this stage to facilitate access throughout the script. This setup ensures that all necessary variables are ready for iterative processing and further calculations beyond the SCF loop.
"""
K, F, D, C, eps = J.transpose(0, 3, 2, 1), np.zeros_like(S), np.zeros_like(S), np.zeros_like(S), np.array(nbf * [0])

"""
This while loop is the SCF loop. Please fill it so it calculates the Fock matrix, solves the Fock equations, builds the density matrix from the coefficients and calculates the energy. You can use all the variables defined above and all the functions in numpy package. The recommended functions are np.einsum and np.linalg.eigh. Part of the calculation will probably be calculation of the inverse square root of a matrix. The numpy package does not conatin a function for this. You can find a library that can do that or you can do it manually. The manual calculation is, of course, preferred.
"""
while abs(E_HF - E_HF_P) > thresh:
    break

"""
In the followng block of code, please calculate the nuclear-nuclear repulsion energy. You should use only the atoms and coords variables. The code can be as short as two lines. The result should be stored in the "VNN" variable.
"""
VNN = 0

# print the results
print("RHF ENERGY: {:.8f}".format(E_HF + VNN))
\end{lstlisting}

\raggedbottom\begin{lstlisting}[language=Python, caption={Integral transform exercise code. The tiling matrices are predefined here with a correct shape, but the student is expected to fill the with the correct expressions. After that, the student should transform the coefficient matrix, core Hamoltonian, Fock matrix and two-electron integrals to the \acrshort{ms} basis. Additionally, orbital energy tensor are also expected to be calculated.}, label=code:int_exercise]
"""
To perform most of the post-HF calculations, we need to transform the Coulomb integrals to the molecular spinorbital basis, so if you don't plan to calculate any post-HF methods, you can end the eercise here. The restricted MP2 calculation could be done using the Coulomb integral in MO basis, but for the sake of subsequent calculations, we enforce here the integrals in the MS basis. The first thing you will need for the transform is the coefficient matrix in the molecular spinorbital basis. To perform this transform using the mathematical formulation presented in the materials, the first step is to form the tiling matrix "P" which will be used to duplicate columns of a general matrix. Please define it here.
"""
P = np.zeros((nbf, 2 * nbf))

"""
Now, please define the spin masks "M" and "N". These masks will be used to zero out spinorbitals, that should be empty.
"""
M, N = np.zeros((nbf, 2 * nbf)), np.zeros((nbf, 2 * nbf))

"""
With the tiling matrix and spin masks defined, please transform the coefficient matrix into the molecular spinorbital basis. The resulting matrix should be stored in the "Cms" variable.
"""
Cms = np.zeros(2 * np.array(C.shape))

"""
For some of the post-HF calculations, we will also need the Hamiltonian and Fock matrix in the molecular spinorbital basis. Please transform it and store it in the "Hms" and "Fms" variable. If you don't plan to calculate the CCSD method, you can skip the transformation of the Fock matrix, as it is not needed for the MP2 and CI calculations.
"""
Hms, Fms = np.zeros(2 * np.array(H.shape)), np.zeros(2 * np.array(H.shape))

"""
With the coefficient matrix in the molecular spinorbital basis available, we can proceed to transform the Coulomb integrals. It is important to note that the transformed integrals will contain twice as many elements along each axis compared to their counterparts in the atomic orbital (AO) basis. This increase is due to the representation of both spin states in the molecular spinorbital basis.
"""
Jms = np.zeros(2 * np.array(J.shape))

"""
The post-HF calculations also require the antisymmetrized two-electron integrals in the molecular spinorbital basis. These integrals are essential for the MP2 and CC calculations. Please define the "Jmsa" tensor as the antisymmetrized two-electron integrals in the molecular spinorbital basis.
"""
Jmsa = np.zeros(2 * np.array(J.shape))

"""
As mentioned in the materials, it is also practical to define the tensors of reciprocal orbital energy differences in the molecular spinorbital basis. These tensors are essential for the MP2 and CC calculations. Please define the "Emss", "Emsd" and "Emst" tensors as tensors of single, double and triple excitation energies, respectively. The configuration interaction will not need these tensors, so you can skip this step if you don't plan to program the CI method. The MP methods will require only the "Emsd" tensor, while the CC method will need both tensors.
"""
Emss, Emsd = np.array([]), np.array([])
\end{lstlisting}

Solution to this exercises can be found in Section \ref{sec:hf_code_solution} and in Section \ref{sec:int_code_solution}.
\chapter{Møller--Plesset Perturbation Theory}

\acrshort{mppt} is a quantum mechanical method used to improve the accuracy of electronic structure calculations within the framework of \acrshort{hf} theory. It involves treating electron-electron correlation effects as a perturbation to the reference \acrshort{hf} wave function. The method is named after its developers, physicists C. Møller and M. S. Plesset. By systematically including higher-order corrections, \acrshort{mppt} provides more accurate predictions of molecular properties compared to the initial \acrshort{hf} approximation.

\section{Theory of the Perturbative Approach}

As for the \acrshort{hf} method, we start with the Schrödinger equation in the form

\begin{equation}
\hat{\mathbf{H}}\ket{\Psi}=E\ket{\Psi}
\end{equation}

where \(\hat{\mathbf{H}}\) is the molecular Hamiltonian operator, \(\ket{\Psi}\) is the molecular wave function, and \(E\) is the total energy of the system. In the Møller--Plesset perturbation theory we write the Hamiltonian operator as

\begin{equation}
\hat{\mathbf{H}}=\hat{\mathbf{H}}^{(0)}+\lambda\hat{\mathbf{H}}^{'}
\end{equation}

where \(\hat{\mathbf{H}}^{(0)}\) is the Hamiltonian used in the \acrshort{hf} method (representing electrons moving in the mean field), \(\lambda\) is a parameter between 0 and 1, and \(\hat{\mathbf{H}}^{'}\) is the perturbation operator representing the missing electron-electron interactions not included in the \acrshort{hf} approximation. We then expand the wavefunction \(\ket{\Psi}\) and total energy \(E\) as a power series in \(\lambda\) as

\begin{align}
\ket{\Psi}&=\ket{\Psi^{(0)}}+\lambda\ket{\Psi^{(1)}}+\lambda^2\ket{\Psi^{(2)}}+\dots \\
E&=E^{(0)}+\lambda E^{(1)}+\lambda^2 E^{(2)}+\dots
\end{align}

and ask, how how does the total energy change with the included terms. After some algebra, we can show that the first order correction to the total energy is zero, the second order correction is given by

\begin{equation}
E_{\mathrm{corr}}^{\mathrm{MP2}}=\sum_{s>0}\frac{H_{0s}^{'}H_{s0}^{'}}{E_0-E_s}
\end{equation}

where \(s\) runs over all doubly excited determinants, \(H_{0s}^{'}\) is the matrix element of the perturbation operator between the \acrshort{hf} determinant and the doubly excited determinant, and \(E_0\) and \(E_s\) are the energies of the reference and doubly excited determinants, respectively.\cite{10.1002/wcms.58,1014569052} We could express all higher-order corrections in a similar way, using only the matrix elements of the perturbation operator and the energies of the determinants. For practical calculations, we apply Slater-Condon rules to evaluate the matrix elements and use the orbital energies obtained from the Hartree-Fock calculation. The expressions for calculation are summarised below.

\section{Implementation of 2nd and 3rd Order Corrections}

Having the antisymmetrized two-electron integrals in the \acrshort{ms} basis and physicists' notation defined in Section \ref{sec:integral_transform}, we can now proceed with the calculation of the correlation energy. The 2nd order correlation energy can be expressed as

\begin{equation}
E_{\mathrm{corr}}^{\mathrm{MP2}}=\frac{1}{4}\sum_{ijab}\frac{\braket{ab||ij}\braket{ij||ab}}{\varepsilon_{ij}^{ab}}
\end{equation}

and the 3rd order correlation energy as

\begin{align}
E_{\mathrm{corr}}^{\mathrm{MP3}}=&\frac{1}{8}\sum_{ijab}\frac{\braket{ab||ij}\braket{cd||ab}\braket{ij||cd}}{\varepsilon_{ij}^{ab}\varepsilon_{ij}^{cd}}+\nonumber \\
&+\frac{1}{8}\sum_{ijab}\frac{\braket{ab||ij}\braket{ij||kl}\braket{kl||ab}}{\varepsilon_{ij}^{ab}\varepsilon_{kl}^{ab}}+\nonumber \\
&+\sum_{ijab}\frac{\braket{ab||ij}\braket{cj||kb}\braket{ik||ac}}{\varepsilon_{ij}^{ab}\varepsilon_{ik}^{ac}}
\end{align}

To calculate the 4th order correction, we would need to write 39 terms, which is not practical. Higher-order corrections are usually not programmed this way, instead, the diagrammatic approach is used.\cite{1014569052,10.1016/0010-4655!73!90016-7,10.1016/0010-4655!73!90017-9}

\section{2nd and 3rd Order Corrections Code Exercise}

Similar to the \acrshort{hf} method, Møller--Plesset perturbation theory can also be implemented in Python. The code exercise below provides a self-contained guide to calculating the \acrfull{mp2} and \acrfull{mp3} correlation energies. This exercise is designed to be appended to your existing \acrshort{hf} implementation, as the \acrshort{mp2} and \acrshort{mp3} methods build on the results of the \acrshort{hf} procedure. You can access the foundational \acrshort{hf} method and integral transformation coding exercise in Section \ref{sec:hf_int_code_exercise}. The exercise is provided in the Listing \ref{code:mp_exercise} below

\raggedbottom\begin{lstlisting}[language=Python, caption={\acrshort{mp2} and \acrshort{mp3} exercise code. The placeholders for the energies are initialized to zero. If you transformed all the necessary integrals in the previous exercise, you should be able to fill the placeholders with correct expressions.}, label=code:mp_exercise]
"""
Since we have everything we need for the MP calculations, we can now calculate the MP2 correlation energy. The result should be stored in the "E_MP2" variable.
"""
E_MP2 = 0

"""
Let's not stop here. We can calculate MP3 correlation energy as well. Please calculate it and store it in the "E_MP3" variable.
"""
E_MP3 = 0

# print the results
print("MP2 ENERGY: {:.8f}".format(E_HF + E_MP2 +       + VNN))
print("MP3 ENERGY: {:.8f}".format(E_HF + E_MP2 + E_MP3 + VNN))
\end{lstlisting}

Solution to this exercise can be found in Section \ref{sec:mp_code_solution}.
\chapter{Configuration Interaction}

\acrshort{ci} is a \acrshort{post-hf}, utilizing a linear variational approach to address the nonrelativistic Schrödinger equation under the Born--Oppenheimer approximation for multi-electron quantum systems. \acrshort{ci} mathematically represents the wave function as a linear combination of Slater determinants. The term ``configuration'' refers to different ways electrons can occupy orbitals, while ``interaction'' denotes the mixing of these electronic configurations or states. \acrshort{ci} computations, however, are resource-intensive, requiring significant CPU time and memory, limiting their application to smaller molecular systems. While \acrshort{fci} considers all possible electronic configurations, making it computationally prohibitive for larger systems, truncated versions like \acrfull{cisd} or \acrfull{cisdt} are more feasible and commonly employed in quantum chemistry studies.

\section{Theoretical Background of General Configuration Interaction}

In \acrshort{ci} theory, we expand the wavefunction \(\ket{\Psi}\) in terms of the \acrshort{hf} reference determinant and its excited configurations as

\begin{equation}
\ket{\Psi}=c_0\ket{\Psi_0}+\left(\frac{1}{1!}\right)^2c_i^a\ket{\Psi_i^a}+\left(\frac{1}{2!}\right)^2c_{ij}^{ab}\ket{\Psi_{ij}^{ab}}+\left(\frac{1}{3!}\right)^2c_{ijk}^{abc}\ket{\Psi_{ijk}^{abc}}+\dots
\end{equation}

where we seek the coefficients \(\mathbf{c}\) that minimize the energy. To determine these coefficients, we construct and diagonalize the Hamiltonian matrix in the basis of these excited determinants. The \acrshort{ci} Hamiltonian matrix \(\mathbf{H}^{\mathrm{CI}}\) is represented as

\begin{equation}\label{eq:ci-hamiltonian}
\mathbf{H}^{\mathrm{CI}}=
\begin{bmatrix}
\braket{\Psi_0|\hat{H}|\Psi_0} & \braket{\Psi_0|\hat{H}|\Psi_i^a} & \braket{\Psi_0|\hat{H}|\Psi_{ij}^{ab}} & \dots \\
\braket{\Psi_i^a|\hat{H}|\Psi_0} & \braket{\Psi_i^a|\hat{H}|\Psi_i^a} & \braket{\Psi_i^a|\hat{H}|\Psi_{ij}^{ab}} & \dots \\
\braket{\Psi_{ia}^{jb}|\hat{H}|\Psi_0} & \braket{\Psi_{ia}^{jb}|\hat{H}|\Psi_1} & \braket{\Psi_{ia}^{jb}|\hat{H}|\Psi_{ia}^{jb}} & \dots \\
\vdots & \vdots & \vdots & \ddots
\end{bmatrix}
\end{equation}

After the Hamiltonian matrix is constructed, we solve the eigenvalue problem

\begin{equation}\label{eq:ci-eigenvalue-problem}
\mathbf{H}^{\mathrm{CI}}\mathbf{C}^{\mathrm{CI}}=\mathbf{C}^{\mathrm{CI}}\bm{\varepsilon}^{\mathrm{CI}}
\end{equation}

where \(\mathbf{C}^{\mathrm{CI}}\) is a matrix of coefficients and \(\bm{\varepsilon}^{\mathrm{CI}}\) is a diagonal matrix of eigenvalues. The lowest eigenvalue gives the ground-state energy, and the corresponding eigenvector provides the coefficients that minimize the energy. The elements of the \acrshort{ci} Hamiltonian matrix are computed using the Slater--Condon rules, summarized in one function as

\begin{equation}\label{eq:slater-condon-rules}
\mathbf{H}_{ij}^{\mathrm{CI}}=
\begin{cases} 
\displaystyle \sum_kH_{kk}^{\mathrm{core},\mathrm{MS}}+\frac{1}{2}\sum_k\sum_l\braket{kl||kl}&D_i=D_j \\
\displaystyle H_{pr}^{\mathrm{core},\mathrm{MS}}+\sum_k\braket{pk||lk}&D_i=\left\lbrace\dotsi p\dotsi\right\rbrace\land D_j=\left\lbrace\dotsi r\dotsi\right\rbrace \\
\displaystyle \vphantom{\sum_k}\braket{pq||rs}&D_i=\left\lbrace\dotsi p\dotsi q\dotsi\right\rbrace\land D_j=\left\lbrace\dotsi r\dotsi s\dotsi\right\rbrace \\
\displaystyle \vphantom{\sum_k}0&\text{otherwise}
\end{cases}
\end{equation}

where \(D_i\) and \(D_j\) are Slater determinants, \(\mathbf{H}^{\mathrm{core},\mathrm{MS}}\) is the core Hamiltonian in the \acrshort{ms} basis, and \(\braket{pk||lk}\) are the antisymmetrized two-electron integrals in \acrshort{ms} basis and physicists' notation. The sums extend over all spinorbitals common between the two determinants. These integrals were previously transformed in Section \ref{sec:integral_transform}. Keep in mind, that to apply the Slater-Condon rules, the determinants must be aligned, and the sign of the matrix elements must be adjusted accordingly, based on the number of permutations needed to align the determinants.

An important caveat in \acrshort{ci} theory is its lack of size-extensivity, which implies that the energy does not scale linearly with the number of electrons. This drawback stems from the fact that the \acrshort{ci} wavefunction is not size-consistent, meaning the energy of a combined system is not simply the sum of the energies of its isolated parts. This limitation restricts the application of \acrshort{ci} mainly to small molecular systems.

\section{Full Configuration Interaction Implementation}

In \acrshort{fci}, we aim to account for all possible electronic configurations within a chosen basis set, offering the most accurate wavefunction representation for the given basis. Although this method yields highly precise electronic structure information, it is computationally intensive. Its cost scales exponentially with both the number of electrons and basis functions, limiting its feasibility to smaller systems.

The \acrshort{fci} process involves constructing all possible Slater determinants for a system. For simplicity, we'll assume that we want to include both singlet and triplet states in our determinant space. The total number of these determinants \(N_D\) can be calculated using binomial coefficients

\begin{equation}
N_D=\binom{n}{k}
\end{equation}

where \(k\) is the total number of electrons, and \(n\) is the total number of spinorbitals. For practical representation, it's useful to describe determinants as arrays of numbers, where each number corresponds to the index of an occupied orbitals. For example, the ground state determinant for a system with 6 electrons can be represented as \(\left\lbrace 0,1,2,3,4,5\right\rbrace\), whereas the determinant \(\left\lbrace 0,1,2,3,4,6\right\rbrace\) represents an excited state with one electron excited from orbital 5 to orbital 6. Using the determinants, the \acrshort{ci} Hamiltonian matrix \eqref{eq:ci-hamiltonian} can be constructed, and the eigenvalue problem \eqref{eq:ci-eigenvalue-problem} can be solved to obtain the ground and excited state energies.

\section{Full Configuration Interaction Code Exercise}

The \acrshort{fci} example builds on the \acrshort{hf} method and demonstrates how to implement a \acrshort{fci} calculation in Python using NumPy. This exercise focuses on generating determinants, constructing the \acrshort{ci} Hamiltonian matrix using Slater--Condon rules, and solving the eigenvalue problem to obtain the ground state energy. The code is designed for educational purposes and is based on prior \acrshort{hf} results that can be done in Section \ref{sec:hf_int_code_exercise}. The exercise is provided in the Listing \ref{code:ci_exercise} below.

\raggedbottom\begin{lstlisting}[language=Python, caption={\acrshort{ci} exercise code. Here, student is expected to calculate the \acrshort{ci} ground state energy. The task here is to fill the  \acrshort{ci} Hamiltonian using the Slater--Condon rules and diagonalize it.}, label=code:ci_exercise]
"""
Since we already calculated the necessary integrals in the MS basis, we can proceed. The next step involves generating determinants. We will store these in a simple list, with each determinant represented by an array of numbers, where each number corresponds to an occupied spinorbital. Since we are programming for Full Configuration Interaction (FCI), we aim to generate all possible determinants. However, should we decide to implement methods like CIS, CID, or CISD, we could easily limit the number of excitations. It is important to remember that for all CI methods, the rest of the code remains unchanged. The only difference lies in the determinants used. Don't overcomplicate this. Generating all possible determinants can be efficiently achieved using a simple list comprehension. I recommend employing the combinations function from the itertools package to facilitate this task.
"""
dets = list()

"""
Now, for your convenince, I define here the CI Hamiltonian.
"""
Hci = np.zeros([len(dets), len(dets)])

"""
Before we begin constructing the Hamiltonian, I recommend defining the Slater-Condon rules. Let's consider that the input for these functions will be an array of spinorbitals, segmented into unique and common ones. A practical approach might be to arrange this 1D array with all unique spinorbitals at the front, followed by the common spinorbitals. This arrangement allows you to easily determine the number of unique spinorbitals based on the rule being applied, meaning you will always know how many entries at the beginning of the array are unique spinorbitals. While you can develop your own method for managing this array, I will proceed under the assumption that the Slater-Condon rules we use will take a single array of spinorbitals and return an unsigned matrix element. The sign of this element will be corrected later in the script. For simplicity and flexibility, I'll define these rules using lambda functions, but you're welcome to expand them into full functions if you prefer.
"""
slater0 = lambda so: 0
slater1 = lambda so: 0
slater2 = lambda so: 0

"""
We can now proceed to filling the CI Hamiltonian. The loop is simple.
"""
for i in range(Hci.shape[0]):
    for j in range(Hci.shape[1]):

        """
        The challenging part of this process is aligning the determinants. In this step, I transfer the contents of the j-th determinant into the "aligned" determinant. It's important not to alter the j-th determinant directly within its original place, as doing so could disrupt the computation of other matrix elements. Instead, we carry out the next steps on the determinant now contained in the "aligned" variable. Additionally, the element sign is defined at this stage. You probably want to leave this unchanged.
        """
        aligned, sign = dets[j].copy(), 1

        """
        Now it's your turn. Please adjust the "aligned" determinant to match the i-th determinant as closely as possible. By "align", I mean you should execute a series of spinorbital swaps to minimize the differences between the "aligned" and the i-th determinant. It's also important to monitor the number of swaps you make, as each swap affects the sign of the determinant, hence the reason for the "sign" variable defined earlier. This task is not straightforward, so don't hesitate to reach out to the authors if you need guidance.
        """
        aligned = aligned

        """
        After aligning, we end up with two matched determinants: "aligned" and "dets[i]". At this point, we can apply the Slater-Condon rules. I suggested earlier that the input for these rules should be an array combining both unique and common spinorbitals. You can prepare this array now. However, if you've designed your Slater-Condon rules to directly accept the determinants instead, you can skip this preparatory step.
        """
        so = list()

        """
        Now, you'll need to assign the matrix element. Start by determining the number of differences between the two determinants. Based on this number, apply the corresponding Slater-Condon rule. Don't forget to multiply the result by the sign to account for any changes due to swaps made during the alignment of the determinants.
        """
        H[i, j] = 0

"""
You can finally solve the eigenvalue problem. Please, assign the correlation energy to the "E_FCI" variable.
"""
E_FCI = 0

# print the results
print("FCI ENERGY: {:.8f}".format(E_HF + E_FCI + VNN))
\end{lstlisting}

Solution to this exercise can be found in Section \ref{sec:ci_code_solution}.
\chapter{Coupled Cluster Theory}

\acrshort{cc} theory is a \acrshort{post-hf} method used in quantum chemistry to achieve highly accurate solutions to the electronic Schrödinger equation, particularly for ground states and certain excited states. It improves upon \acrshort{hf} by incorporating electron correlation effects through a systematic inclusion of excitations (singles, doubles, triples, etc.) from a reference wavefunction, usually the \acrshort{hf} wavefunction. The method uses an exponential ansatz to account for these excitations, leading to a size-consistent and size-extensive approach, making it one of the most accurate methods available for small to medium-sized molecular systems.

Within \acrshort{cc} theory, specific truncations are often applied to manage computational cost. The \acrshort{ccd} method considers only double excitations, capturing electron correlation more effectively than simpler methods like \acrshort{hf}, but at a lower computational expense than higher-level methods. \acrshort{ccsd} extends this approach by including both single and double excitations, offering greater accuracy, particularly for systems where single excitations play a significant role. \acrshort{ccsd} is widely used due to its balance between accuracy and computational feasibility, making it a reliable choice for many chemical systems. \#\# \acrshort{cc} Formalism

In the \acrshort{cc} formalism, we write the total wavefunction in an exponential form as

\begin{equation}
\ket{\Psi}=e^{\hat{\mathbf{T}}}\ket{\Psi_0}
\end{equation}

where \(\ket{\Psi_0}\) is the reference wavefunction, usually the \acrshort{hf} wavefunction, and \(\hat{T}\) is the cluster operator that generates excitations from the reference wavefunction. The cluster operator is defined as

\begin{equation}
\hat{\mathbf{T}}=\hat{\mathbf{T}}_1+\hat{\mathbf{T}}_2+\hat{\mathbf{T}}_3+\dots
\end{equation}

where \(\hat{\mathbf{T}}_1\) generates single excitations, \(\hat{\mathbf{T}}_2\) generates double excitations, and so on. For example

\begin{equation}
\hat{\mathbf{T}}_1\ket{\Psi_0}=\left(\frac{1}{1!}\right)^2t_i^a\ket{\Psi_i^a}
\end{equation}

where \(t_i^a\) are the single excitation amplitudes. These amplitudes are just expansion coefficients that determine the contribution of each excitation to the total wavefunction. In the context of configuration interaction, we denoted these coefficients as \(c_i^a\). Now that we have the total wavefunction, we want to solve the Schrödinger equation

\begin{equation}
\hat{\mathbf{H}}\ket{\Psi}=E\ket{\Psi}
\end{equation}

where \(\hat{H}\) is the molecular Hamiltonian operator, \(E\) is the total energy of the system, and \(\ket{\Psi}\) is the total wavefunction. In the \acrshort{cc} theory, we usually rewrite the Schrödinger equation in the exponential form as

\begin{equation}
e^{-\hat{\mathbf{T}}}\hat{\mathbf{H}}e^{\hat{\mathbf{T}}}\ket{\Psi_0}=E\ket{\Psi_0}
\end{equation}

because we can then express the \acrshort{cc} energy as

\begin{equation}
E=\braket{\Psi_0|e^{-\hat{\mathbf{T}}}\hat{\mathbf{H}}e^{\hat{\mathbf{T}}}|\Psi_0}
\end{equation}

taking advantage of the exponential form of the wavefunction. We could then proceed to express the total energy for various \acrshort{cc} methods like \acrshort{ccd} and \acrshort{ccsd}, but the equations would be quite lengthy. Instead, we will leave the theory here and proceed to the actual calculations. One thing to keep in mind is that the \acrshort{cc} equations are nonlinear and require iterative solution methods to obtain the final amplitudes.

\section{Implementation of Truncated Coupled Cluster Methods}

We will not go into the details here, but we will provide the final expressions for the \acrshort{ccd} and \acrshort{ccsd} methods.\cite{10.1063/1.460620} The \acrshort{ccd} and \acrshort{ccsd} methods are the most commonly used \acrshort{cc} methods, and they are often used as benchmarks for other methods. All we need for the evaluation of the expressions below are the two-electron integrals in the \acrshort{ms} basis and physicists' notation, Fock matrix in the \acrshort{ms} basis and the orbital energy tensors obtained from the \acrshort{hf} calculation. All these transformations are already explained in Section \ref{sec:integral_transform}. The expressions for the \acrshort{ccd} can be written as

\begin{equation}
E_{\text{CCD}}=\frac{1}{4}\braket{ij||ab}t_{ij}^{ab}
\end{equation}

where the double excitation amplitudes \(t_{ij}^{ab}\) are determined by solving the \acrshort{ccd} amplitude equation. The \acrshort{ccd} amplitude equations are given by

\begin{align}
t_{ij}^{ab}=&\braket{ab||ij}+\frac{1}{2}\braket{ab||cd}t_{cd}^{ij}+\frac{1}{2}\braket{kl||ij}t_{ab}^{kl}+\hat{P}_{(a/b)}\hat{P}_{(i/j)}\braket{ak||ic}t_{cb}^{ij}-\nonumber \\
&-\frac{1}{2}\hat{P}_{(a/b)}\braket{kl||cd}t_{ac}^{ij}t_{bd}^{kl}-\frac{1}{2}\hat{P}_{(i/j)}\braket{kl||cd}t_{ab}^{ik}t_{cd}^{jl}+\nonumber \\
&+\frac{1}{4}\braket{kl||cd}t_{cd}^{ij}t_{ab}^{kl}+\hat{P}_{(i/j)}\braket{kl||cd}t_{ac}^{ik}t_{bd}^{jl}
\end{align}

where \(\hat{P}_{(a/b)}\) and \(\hat{P}_{(i/j)}\) are permutation operators that ensure the correct antisymmetry of the amplitudes. The \acrshort{ccsd} energy expression is given by

\begin{equation}
E_{\text{CCSD}}=F_{ia}^{\mathrm{MS}}t_a^i+\frac{1}{4}\braket{ij||ab}t_{ij}^{ab}+\frac{1}{2}\braket{ij||ab}t_{i}^{a}t_{b}^{j}
\end{equation}

where the single and double excitation amplitudes \(t_a^i\) and \(t_{ij}^{ab}\) are determined by solving the \acrshort{ccsd} amplitude equations. To simplify the notation a little bit, we define the the \(\mathscr{F}\) and \(\mathscr{W}\) intermediates as

\begin{align}
\mathscr{F}_{ae}=&\left(1-\delta_{ae}\right)F_{ae}-\frac{1}{2}F_{me}t_m^a+t_m^f\braket{ma||fe}-\frac{1}{2}\tilde{\tau}_{mn}^{af}\braket{mn||ef} \\
\mathscr{F}_{mi}=&\left(1-\delta_{mi}\right)F_{mi}+\frac{1}{2}F_{me}t_i^e+t_n^e\braket{mn||ie}+\frac{1}{2}\tilde{\tau}_{in}^{ef}\braket{mn||ef} \\
\mathscr{F}_{me}=&F_{me}+t_n^f\braket{mn||ef} \\
\mathscr{W}_{mnij}=&\braket{mn||ij}+\hat{P}_{(i/j)}t_j^e\braket{mn||ie}+\frac{1}{4}\tau_{ij}^{ef}\braket{mn||ef} \\
\mathscr{W}_{abef}=&\braket{ab||ef}-\hat{P}_{(a/b)}t_m^b\braket{am||ef}+\frac{1}{4}\tau_{mn}^{ab}\braket{mn||ef} \\
\mathscr{W}_{mbej}=&\braket{mb||ej}+t_j^f\braket{mb||ef}-t_n^b\braket{mn||ej}-\left(\frac{1}{2}t_{jn}^{fb}+t_j^ft_n^b\right)\braket{mn||ef}
\end{align}

and two-particle excitation operators as

\begin{align}
\tilde{\tau}_{ij}^{ab}=&t_{ij}^{ab}+\frac{1}{2}\left(t_i^at_j^b-t_i^bt_j^a\right) \\
\tau_{ij}^{ab}=&t_{ij}^{ab}+t_i^at_j^b-t_i^bt_j^a
\end{align}

The \acrshort{ccsd} single excitations amplitude equations are then given by

\begin{align}
t_i^a=&F_{ai}^{\mathrm{MS}}+t_i^e\mathscr{F}_{ae}-t_m^a\mathscr{F}_{mi}t_{im}^{ae}\mathscr{F}_{me}-t_n^f\braket{na||if}-\nonumber-\frac{1}{2}t_{im}^{ef}\braket{ma||ef}- \\
&-\frac{1}{2}t_{mn}^{ae}\braket{nm||ei}
\end{align}

and the \acrshort{ccsd} double excitations amplitude equations are given by

\begin{align}
t_{ij}^{ab}=&\braket{ab||ij}+\hat{P}_{(a/b)}t_{ij}^{ae}\left(\mathscr{F}_{be}-\frac{1}{2}t_m^b\mathscr{F}_{ae}\right)-\hat{P}_{(i/j)}t_{im}^{ab}\left(\mathscr{F}_{mi}+\frac{1}{2}t_j^e\mathscr{F}_{me}\right)+\nonumber \\
&+\frac{1}{2}\tau_{mn}^{ab}\mathscr{W}_{mnij}+\frac{1}{2}\tau_{ij}^{ef}\mathscr{W}_{abef}+\hat{P}_{(i/j)}\hat{P}_{(a/b)}\left(t_{im}^{ae}\mathscr{W}_{mbej}-t_i^et_m^a\braket{mb||ej}\right)+\nonumber \\
&+\hat{P}_{(i/j)}t_i^e\braket{ab||ej}-\hat{P}_{(a/b)}t_m^a\braket{mb||ij}
\end{align}

The \acrshort{ccsd} amplitude equations are, again, nonlinear and require iterative solution methods to obtain the final amplitudes. The initial guess for the amplitudes is often set to zero, and the equations are solved iteratively until convergence is achieved.

\section{Coupled Cluster Singles and Doubles Code Exercise}

After completing the \acrshort{hf} implementation in Section \ref{sec:hf_int_code_exercise}, you can proceed with coding the \acrshort{ccsd} exercise, which builds on the \acrshort{hf} results. The exercise is provided in the Listing \ref{code:cc_exercise} below.

\raggedbottom\begin{lstlisting}[language=Python, caption={\acrshort{ccsd} exercise code. The energy and amplitudes are initialized with default values. The student is expected to fill the loop for the calculation of the excitation amplitudes and ground state energy. After the self-consistency is achieved the result is automatically printed.}, label=code:cc_exercise]
"""
We also have everything we need for the CC calculations. In this exercise, we will calculate the CCSD energy. Since the calculation will be iterative, I define here the CCSD energy as zero, the "E_CCSD_P" variable will be used to monitor convergence.
"""
E_CCSD, E_CCSD_P = 0, 1

"""
The first step of the calculation is to define the "t1" and "t2" amplitudes. These arrays can be initialized as zero arrays with the appropriate dimensions. I will leave this task to you.
"""
t1, t2 = np.array([]), np.array([])

"""
Now for the more complicated part. The CCSD calculation is iterative, and the convergence criterion is set by the "thresh" variable. The while loop should be filled with the appropriate calculations. The calculation of the "t1" and "t2" amplitudes is the most challenging part of the CCSD calculation. After convergence, the "E_CCSD" variable should store the final CCSD energy.
"""
while abs(E_CCSD - E_CCSD_P) > thresh:
    break

# print the CCSD energy
print("CCSD ENERGY: {:.8f}".format(E_HF + E_CCSD + VNN))
\end{lstlisting}

Solution to this exercise can be found in Section \ref{sec:cc_code_solution}.
