


\vspace{5cm}
\section*{SKRIPTA}
H\"{u}ckelova metoda (dále jen HMO, z~angl. \textit{H\"{u}ckel Molecular Orbitals}) je historicky první a nejjednodušší semiempirická metoda. Pro menší molekuly je pro použití této metody potřeba doslova jen tužka a papír. Soustředíme se pouze na určitou skupinu elektronů, energeticky dobře oddělenou od ostatních elektronů. Typickým příkladem jsou vazebné elektrony $\pi$ v~konjugovaných uhlovodících, tedy elektrony v~molekulových orbitalech složených z~atomových orbitalů p -- tento případ budeme nadále uvažovat. HMO ale také dobře popisuje elektronovou strukturu ve valenčním pásu alkalických kovů. 
 
 
Předpokládáme, že geometrii studované molekuly známe. Hamiltonián v H\"{u}ckelově metodě má tvar
\begin{equation}
\hat{H}_{\mathrm{HMO}}= \sum \hat{h}_{i, \mathrm{eff}}^\pi
\label{rov:semiemp:ham}
\end{equation}
kde $\hat{h}_{i, \mathrm{eff}}^\pi$ je jednoelektronový operátor elektronu $\pi$, který v~sobě obsahuje jak kinetickou energii, tak interakci s~ostatními elektrony.
Z~tvaru hamiltoniánu coby součtu jednoelektronových členů je zřejmé, že elektrony jsou zde považovány za nezávislé (to je rozdíl oproti metodě HF, ve které je nezávislost elektronů vyjádřena pouze tvarem vlnové funkce). 
Konkrétní tvar těchto efektivních elektronových hamiltoniánů není specifikován, neboť, jak uvidíme dále, vlastně ani není potřeba -- v~teorii nakonec vystupují pouze integrály, kterým dáváme konkrétní číselnou hodnotu. 

Dalším krokem je výběr báze. V~případě metody HMO to budou orbitaly p$_z$ všech uhlíkových atomů. Molekulový orbital pak hledáme v~běžném rozvoji typu MO-LCAO
\begin{equation}
\phi= \sum_i c_i \chi_i
\label{rov:HM_MO}
\end{equation}
Konkrétní matematický tvar těchto orbitalů nás ale opět nezajímá. Rozvojem do báze převedeme problém nalezení vlnové funkce na jednodušší problém nalezení rozvojových koeficientů $c_i$. Pokud rozvoj \eqref{rov:HM_MO} dosadíme do Schr\"{o}dingerovy rovnici s~hamiltoniánem \eqref{rov:semiemp:ham} a aplikujeme variační princip, dostaneme soustavu sekulárních rovnic, např. pro čtyři elektrony $\pi$ v~butadienu.
\begin{equation}
\begin{split}
\left(H_{11}-\varepsilon S_{11}\right)c_1 + \left(H_{12}-\varepsilon S_{12}\right)c_2 + \left(H_{13}-\varepsilon S_{13}\right)c_3 + \left(H_{14}-\varepsilon S_{14}\right)c_4 = 0 \\ 
\left(H_{21}-\varepsilon S_{21}\right)c_1 + \left(H_{22}-\varepsilon S_{22}\right)c_2 + \left(H_{22}-\varepsilon S_{22}\right)c_3 + \left(H_{24}-\varepsilon S_{24}\right)c_4 = 0 \\ 
\left(H_{31}-\varepsilon S_{31}\right)c_1 + \left(H_{32}-\varepsilon S_{32}\right)c_2 + \left(H_{33}-\varepsilon S_{33}\right)c_3 + \left(H_{34}-\varepsilon S_{34}\right)c_4 = 0 \\ 
\left(H_{41}-\varepsilon S_{41}\right)c_1 + \left(H_{42}-\varepsilon S_{42}\right)c_2 + \left(H_{43}-\varepsilon S_{43}\right)c_3 + \left(H_{44}-\varepsilon S_{44}\right)c_4 = 0 \\  
\end{split}
\label{rov:sekularni}
\end{equation}
%
Z~podmínky pro netriviální řešení této soustavy pak platí
\begin{equation}
|\mathbb{H}-\varepsilon \mathbb{S}|=0 
\label{rov:HM_det}
\end{equation}

Pro molekulu butadienu se elektronová báze skládá ze čtyř orbitalů p$_z$ (viz obrázek \ref{obr:HMbutadien} ukazující možné kombinace orbitalů p na čtyřech atomech uhlíku) a sekulární determinant má tvar
\begin{equation}
\begin{vmatrix}
H_{11}-\varepsilon S_{11} & H_{21}-\varepsilon S_{21} & H_{31}-\varepsilon S_{31} & H_{41}-\varepsilon S_{41}  \\
H_{12}-\varepsilon S_{12} & H_{22}-\varepsilon S_{22} & H_{32}-\varepsilon S_{32} & H_{42}-\varepsilon S_{42}  \\
H_{13}-\varepsilon S_{13} & H_{23}-\varepsilon S_{23} & H_{33}-\varepsilon S_{33} & H_{43}-\varepsilon S_{43}  \\
H_{14}-\varepsilon S_{14} & H_{24}-\varepsilon S_{24} & H_{34}-\varepsilon S_{34} & H_{44}-\varepsilon S_{44}
\end{vmatrix}
= 0
\end{equation}



Nyní bychom měli spočítat integrály $H_{ij}$ a $S_{ij}$, které v~principu závisí na geometrii systému.
V~rámci H\"{u}ckelovy aproximace na ně ale pohlížíme jako na parametry metody
a naložíme s~nimi takto
\begin{itemize}
\item Zanedbáme překryvové integrály mezi AO na různých atomech, tj. $S_{ij}=\delta_{ij}$.
\item Integrály $H_{ii}$ položíme rovny parametru $\alpha$, těmto integrálům říkáme coulombovské integrály.
\item Integrály $H_{ij}$ položíme rovny nule, pokud orbitaly nepatří sousedním atomům.
Nenulové integrály mezi sousedními atomy položíme rovny parametru $\beta$. Nazýváme je rezonančními integrály.
\end{itemize}
Takto zjednodušené rovnice poté vyřešíme. Nejprve najdeme vlastní čísla přes sekulární determinant a následně určíme koeficienty $c_i$. V~případě butadienu vypadá H\"{u}ckelův sekulární determinant takto
\begin{equation}
\begin{vmatrix}
\alpha-\varepsilon & \beta & 0 & 0  \\
\beta &\alpha-\varepsilon & \beta & 0  \\
0 &\beta &\alpha-\varepsilon & \beta  \\
0 & 0 & \beta &\alpha-\varepsilon  
\end{vmatrix}
= 
\begin{vmatrix}
x & 1 & 0 & 0  \\
1 & x & 1 & 0  \\
0 & 1 & x & 1  \\
0 & 0 & 1 & x  
\end{vmatrix}
\end{equation}
Determinant jsme vydělili $\beta$ a zavedli substituci $x=\frac{\alpha-\varepsilon}{\beta}$.
Nyní rozvojem determinantu podle prvního řádku dostaneme
\begin{equation}
x
\begin{vmatrix}
x & 1 & 0 \\
1 & x & 1 \\
0 & 1 & x \\
\end{vmatrix}
-
\begin{vmatrix}
x & 1 & 0 \\
1 & x & 1 \\
0 & 1 & x \\
\end{vmatrix}
=x^4-3x^2+1=0  
\end{equation}
Tuto kvartickou rovnici lze naštěstí převést na kvadratickou pomocí substituce $y=x^2$
\begin{equation}
y^2-3y+1=0,\quad y_1=2{,}62,\quad y_2=0{,}382 
\end{equation}
a tedy
\begin{equation}
x_{1,2}=\pm 1{,}62 \quad
x_{3,4}=\pm 0{,}62
\end{equation}
Z~toho vyplývají následující hodnoty energií
\begin{equation}
\begin{split}
\varepsilon_1 &= \alpha+1{,}62\beta \\
\varepsilon_2 &= \alpha+0{,}62\beta \\
\varepsilon_3 &= \alpha-0{,}62\beta \\
\varepsilon_4 & = \alpha-1{,}62\beta 
\end{split}
\label{rov:HM2}
\end{equation}

Celkovou energii můžeme vypočítat jako součet orbitálních energií (na rozdíl od metody HF!).
Jak ale získáme parametry $\alpha$ a $\beta$? Vyjdeme z~experimentálních dat. Parametr $\alpha$ má jednoduchou fyzikální interpretaci -- jedná se o~ionizační energii atomového orbitalu $p_z$, která je experimentálně známá. Důležitější je parametr $\beta$, který určuje energetické rozdíly mezi molekulovými orbitaly. Můžeme jej získat v~zásadě dvěma způsoby
\begin{enumerate}
\item \textbf{Ze spektroskopických dat.} Parametr zvolíme tak, aby nám seděly zvolené elektronové přechody (podrobnosti viz níže), $\beta = \SI{-3{,}48}{\eV}$.
\item \textbf{Z~termochemických dat.} K~tomu se využívá konceptu delokalizační energie dvojných vazeb. Tuto energii můžeme experimentálně naměřit například pomocí hydrogenačního tepla. Její teoretický výpočet v~rámci HMO si ukážeme níže. Pro tyto účely se typicky používá hodnota $\beta = \SI{-0{,}78}{\ev} = \SI{-75}{\kJ\per\mol}$. Je to podstatně odlišné číslo, což ukazuje na omezení HMO.
\end{enumerate}

\subsection{Vlnové funkce v~rámci HMO}
Abychom získali vlnovou funkci orbitalu s~nejnižší energií, musíme dosadit hodnoty energií ze vztahu \eqref{rov:HM2} zpět do sekulárních rovnic. Pro butadien bychom řešili sadu čtyř rovnic
\begin{equation}
\begin{split}
c_1(\alpha -E)+c_2\beta +0 +0 &=0 \\ 
c_1\beta + c_2(\alpha-E)+c_3\beta + 0 &= 0 \\
0 + c_2\beta + c_3(\alpha-E)_+ c_4\beta &= 0 \\
0 + 0 + c_3\beta+ c_4(\alpha -E) &= 0
\end{split}
\end{equation}
Všechny rovnice můžeme vydělit $\beta$ a můžeme použít substituci
$x = \frac{\alpha-E}{\beta}.$
Rovnice se zjednoduší na
\begin{equation}
\begin{split}
c_1x+c_2 &= 0 \\
c_1+c_2x+c_3 &= 0 \\
c_2 + c_3x+c_4 &= 0 \\
c_3 + c_4x &= 0
\end{split}
\end{equation}
Po několika matematických úpravách dostaneme následující vztahy mezi koeficienty
\begin{equation}
\begin{split}
c_2 &= c_3 \\
c_1 &= c_4 \\
c_2 &= 1{,}62c_1
\end{split}
\end{equation}
Abychom získali numerické hodnoty jednotlivých koeficientů, musíme použít normalizační podmínku
\begin{equation}
c_1^2 + c_2^2 + c_3^2 + c_4^2 = c_1^2+(1{,}62c_1)^2+(1{,}62c_1)^2+c_1^2 = 1
\end{equation}
Řešením je $c_1=0{,}37$ a výsledná vlnová funkce má tvar
\begin{equation}
\phi_1 = 0{,}37\chi_1+0{,}60\chi_2+0{,}60\chi_3+0{,}37\chi_4 
\end{equation}

Vlnovou funkci orbitalu s~nejnižší energií pak můžeme schematicky znázornit následujícím způsobem


Úplně analogicky bychom postupovali pro další orbitaly. Jejich vlnové funkce můžeme napsat jako
\begin{equation}
\begin{split}
\phi_2 = 0{,}60\chi_1+0{,}37\chi_2-0{,}37\chi_3-0{,}60\chi_4 \\
\phi_3 = 0{,}60\chi_1-0{,}37\chi_2-0{,}37\chi_3+0{,}60\chi_4 \\
\phi_4 = 0{,}37\chi_1-0{,}60\chi_2+0{,}60\chi_3-0{,}37\chi_4 
\end{split}
\end{equation}
%
Vlnové funkce orbitalů pak můžeme schematicky znázornit takto

 
\subsection{Aplikace HMO: Výpočet excitační energie}

Jak můžeme pomocí HMO spočítat excitační energii? Excitujeme jeden elektron z~obsazeného do neobsazeného orbitalu (viz obrázek \ref{obr:HMbuta}) a potom spočítáme celkovou energii a od ní odečteme celkovou energii základního stavu. Pokud chceme spočítat nejnižší excitační energii butadienu, přesuneme jeden elektron z~orbitalu 2 do orbitalu 3. Výsledná energie potom bude
\begin{equation}
\Delta E = \varepsilon_3-\varepsilon_2 = -1{,}24 \beta = h\nu 
\end{equation}

Vlnová délka excitujícího záření vyjde \SI{287}{\nm} ($\beta = -\SI{3{,}48}{\eV}$), zatímco experimentální hodnota je kolem \SI{200}{\nm}.
Soulad to není ideální, ale není ani tragický, na to že jsme jej získali tak jednoduše. HMO také správně zachycuje experimentální fakt, že excitační energie se snižuje se zvyšující se délkou řetězce obsahující konjugované dvojné vazby.



\subsection{Aplikace HMO: Výpočet delokalizační energie}

Z~organické chemie si ještě možná vzpomeneme, že konjugované dvojné vazby mají jiné vlastnosti než vazby izolované.
Tuto delokalizaci můžeme vyjádřit pomocí poněkud nefyzikální veličiny delokalizační energie.

Pokud by se dvojné vazby vzájemně neovlivňovaly, tak by mělo platit, že  energie
\textit{n}-krát konjugovaného systému by se rovnala \textit{n}-násobku energie ethenu, který má dvojnou vazbu jen jednu. Tato rovnost ale neplatí a rozdíl je právě delokalizační energie. 

Pojďme si vše ukázat na příkladě butadienu, jehož energii již známe. Musíme tedy dopočítat energii molekuly ethenu. Pro ten vychází následující sekulární determinant 
\begin{equation}
\begin{vmatrix}
\alpha - \varepsilon & \beta \\
\beta & \alpha - \varepsilon \\
\end{vmatrix}
=0
\end{equation}
z~čehož
$$
(\alpha-\varepsilon)^2=\beta^2
$$
$$
\alpha - \varepsilon = \pm \beta
$$
$$
\varepsilon_{1,2}= \alpha\pm \beta 
$$
Pro butadien poté vychází delokalizační energie
\begin{equation}
E_{\mathrm{D}} = E_{\mathrm{butadien}} - 2E_{\mathrm{ethylen}}=2(\alpha + 1{,}62\beta)+ 2(\alpha + 0{,}62\beta) - 4(\alpha + \beta) =0{,}48\beta = -\SI{36}{\kJ\per\mol}
\end{equation}
%
Experimentálně můžeme míru delokalizace určit pomocí hydrogenačních tepel butadienu a ethenu.
Hydrogenační teplo ethenu (tj. entalpie reakce ethen + H$_2 \xrightarrow{}$ ethan) je
\SI{-136{,}94}{\kJ\per\mol}, zatímco hydrogenační teplo butadienu je \SI{-240{,}0}{\kJ\per\mol}. Experimentální delokalizační energie je tedy
\begin{equation}
E_{\mathrm{D}}^{\mathrm{exp}}=2\cdot136{,}94-240{,}0 = \SI{33{,}88}{\kJ\per\mol}
\end{equation}
\noindent což je v~dobrém souladu s~námi vypočtenou hodnotou.

\subsection{Stabilita konjugovaných cyklů  a aromaticita}

Asi největším úspěchem HMO bylo vysvětlení aromaticity cyklických molekul. Na obrázku \ref{obr:HMcykly} jsou znázorněny energetické hladiny cyklických systémů typu C$_n$H$_n$. Parametr $\alpha$ představuje energii elektronu v~samostatném atomu uhlíku. Jestliže má elektron energii nižší než  $\alpha$, interakce s~okolními atomy vede k~poklesu energie. Jinými slovy, dochází ke vzniku chemické vazby. Energie větší než $\alpha$ indikuje destabilizaci. Pohled na diagram přitom pěkně ukazuje, odkud se vzalo známé  H\"{u}ckelovo pravidlo $4n+2$. Je totiž hned patrné, že

\begin{itemize}
\item molekuly s~4$n$+2 elektrony jsou stabilní,
\item molekuly s~4$n$+1 elektrony představují radikály,
\item molekuly s~4$n$ elektrony představují biradikály.
\end{itemize}



\textbf{Řešení:} Cyklopropenyl má tři elektrony v~orbitalech p$_z$. Jednoduché H\"{u}ckelovo pravidlo $4n+2$ by bylo splněno pouze pro $n=\frac{1}{4}$, není to tedy celé číslo, a proto molekula není aromatická. U~cyklopropenylového kationtu je naopak pravidlo splněno, protože molekula má jen 2 elektrony $\pi$. V~tomto případě je $n=0$ a molekula je aromatická a stabilní. Naopak cyklopropenylový anion má počet $\pi$ elektronů roven 4, proto aromatický ani stabilní není. 

\begin{ourbox}{Příběh ferrocenu}
Podle HMO by měl být stabilní pětičlenný cyklus se záporným nábojem, tj. měl by existovat cyklopentadienylový anion. Ve třicátých letech, kdy Erich H\"{u}ckel (ponoukán svým bratrem Walterem, který byl organickým chemikem) vytvořil svou teorii, žádný takový anion znám nebyl. Teprve v~padesátých letech se podařilo syntetizovat ferrocen, sandwichový komplex železnatých iontů a cyklopentadienylových aniontů. 


Takže tento anion existuje a je dokonce velmi stabilní. Ferrocen má dnes svou roli mimo jiné v~elektrochemii, kde se používá jako jedna z~referenčních elektrod. 
\end{ourbox}