\section*{Hückel theory}

\lstset{style=mystyle}
\lstinputlisting[caption=Solution: quantum dynamics in real-time,language=Python]{codes/solutions/huckel.py}

\section*{\texorpdfstring{Hartree--Fock Method\label{sec:hf_code_solution}}{Hartree--Fock Method}}

\raggedbottom\begin{lstlisting}[language=Python, caption={\acrshort{hf} method exercise code solution.}, label=code:hf_solution]
# energies, number of occupied and virtual orbitals and the number of basis functions
E_HF, E_HF_P, VNN, nbf, nocc = 0, 1, 0, S.shape[0], sum(atoms) // 2; nvirt = nbf - nocc

# exchange integrals and the guess density matrix
K, D = J.transpose(0, 3, 2, 1), np.zeros((nbf, nbf))

# Fock matrix, coefficient matrix and orbital energies initialized to zero
F, C, eps = np.zeros((nbf, nbf)), np.zeros((nbf, nbf)), np.zeros((nbf))

# the X matrix which is the inverse of the square root of the overlap matrix
SEP = np.linalg.eigh(S); X = SEP[1] @ np.diag(1 / np.sqrt(SEP[0])) @ SEP[1].T

# the scf loop
while abs(E_HF - E_HF_P) > args.threshold:

    # build the Fock matrix
    F = H + np.einsum("ijkl,ij->kl", J - 0.5 * K, D, optimize=True)

    # solve the Fock equations
    eps, C = np.linalg.eigh(X @ F @ X); C = X @ C

    # build the density from coefficients
    D = 2 * np.einsum("ij,kj->ik", C[:, :nocc], C[:, :nocc])

    # save the previous energy and calculate the current electron energy
    E_HF_P, E_HF = E_HF, 0.5 * np.einsum("ij,ij->", D, H + F, optimize=True)

# calculate nuclear-nuclear repulsion
for i, j in ((i, j) for i, j in it.product(range(natoms), range(natoms)) if i != j):
    VNN += 0.5 * atoms[i] * atoms[j] / np.linalg.norm(coords[i, :] - coords[j, :])

# print the results
print("    RHF ENERGY: {:.8f}".format(E_HF + VNN))
\end{lstlisting}

\section*{\texorpdfstring{Integral Transform\label{sec:int_code_solution}}{Integral Transform}}

\raggedbottom\begin{lstlisting}[language=Python, caption={Integral transform exercise code solution.}, label=code:int_solution]
# define the occ and virt spinorbital slices shorthand
o, v = slice(0, 2 * nocc), slice(2 * nocc, 2 * nbf)

# define the tiling matrix for the Jmsa coefficients and energy placeholders
P = np.array([np.eye(nbf)[:, i // 2] for i in range(2 * nbf)]).T

# define the spin masks
M = np.repeat([1 - np.arange(2 * nbf, dtype=int) % 2], nbf, axis=0)
N = np.repeat([    np.arange(2 * nbf, dtype=int) % 2], nbf, axis=0)

# tile the coefficient matrix, apply the spin mask and tile the orbital energies
Cms, epsms = np.block([[C @ P], [C @ P]]) * np.block([[M], [N]]), np.repeat(eps, 2)

# transform the core Hamiltonian and Fock matrix to the molecular spinorbital basis
Hms = np.einsum("ip,ij,jq->pq", Cms, np.kron(np.eye(2), H), Cms, optimize=True)
Fms = np.einsum("ip,ij,jq->pq", Cms, np.kron(np.eye(2), F), Cms, optimize=True)

# transform the coulomb integrals to the MS basis in chemists' notation
Jms = np.einsum("ip,jq,ijkl,kr,ls->pqrs",
    Cms, Cms, np.kron(np.eye(2), np.kron(np.eye(2), J).T), Cms, Cms, optimize=True
);

# antisymmetrized two-electron integrals in physicists' notation
Jmsa = (Jms - Jms.swapaxes(1, 3)).transpose(0, 2, 1, 3)

# tensor epsilon_i^a
Emss = epsms[o] - epsms[v, None]

# tensor epsilon_ij^ab
Emsd = epsms[o] + epsms[o, None] - epsms[v, None, None] - epsms[v, None, None, None]
\end{lstlisting}

\section*{\texorpdfstring{2nd and 3rd Order Perturbative Corrections\label{sec:mp_code_solution}}{2nd and 3rd Order Perturbative Corrections}}

\raggedbottom\begin{lstlisting}[language=Python, caption={\acrshort{mp2} and \acrshort{mp3} exercise code solution.}, label=code:mp_solution]
# energy containers
E_MP2, E_MP3 = 0, 0

# calculate the MP2 correlation energy
if args.mp2 or args.mp3:
    E_MP2 += 0.25 * np.einsum("abij,ijab,abij",
        Jmsa[v, v, o, o], Jmsa[o, o, v, v], Emsd**-1, optimize=True
    )
    print("    MP2 ENERGY: {:.8f}".format(E_HF + E_MP2 + VNN))

# calculate the MP3 correlation energy
if args.mp3:
    E_MP3 += 0.125 * np.einsum("abij,cdab,ijcd,abij,cdij",
        Jmsa[v, v, o, o], Jmsa[v, v, v, v], Jmsa[o, o, v, v], Emsd**-1, Emsd**-1,
        optimize=True
    )
    E_MP3 += 0.125 * np.einsum("abij,ijkl,klab,abij,abkl",
        Jmsa[v, v, o, o], Jmsa[o, o, o, o], Jmsa[o, o, v, v], Emsd**-1, Emsd**-1,
        optimize=True
    )
    E_MP3 += 1.000 * np.einsum("abij,cjkb,ikac,abij,acik",
        Jmsa[v, v, o, o], Jmsa[v, o, o, v], Jmsa[o, o, v, v], Emsd**-1, Emsd**-1,
        optimize=True
    )
    print("    MP3 ENERGY: {:.8f}".format(E_HF + E_MP2 + E_MP3 + VNN))
\end{lstlisting}

\section*{\texorpdfstring{Full Configuration Interaction\label{sec:ci_code_solution}}{Full Configuration Interaction}}

\raggedbottom\begin{lstlisting}[language=Python, caption={\acrshort{ci} exercise code solution.}, label=code:ci_solution]
# generate the determiants
dets = [np.array(det) for det in it.combinations(range(2 * nbf), 2 * nocc)]

# define the CI Hamiltonian
Hci = np.zeros([len(dets), len(dets)])

# define the Slater-Condon rules, "so" is an array of unique and common spinorbitals
slater0 = lambda so: (
    sum(np.diag(Hms)[so]) + sum([0.5 * Jmsa[m, n, m, n] for m, n in it.product(so, so)])
)
slater1 = lambda so: (
    Hms[so[0], so[1]] + sum([Jmsa[so[0], m, so[1], m] for m in so[2:]])
)
slater2 = lambda so: (
    Jmsa[so[0], so[1], so[2], so[3]]
)

# filling of the CI Hamiltonian
for i in range(0, Hci.shape[0]):
    for j in range(i, Hci.shape[1]):

        # aligned determinant and the sign
        aligned, sign = dets[j].copy(), 1

        # align the determinant "j" to "i" and calculate the sign
        for k in (k for k in range(len(aligned)) if aligned[k] != dets[i][k]):
            while len(l := np.where(dets[i] == aligned[k])[0]) and l[0] != k:
                aligned[[k, l[0]]] = aligned[[l[0], k]]; sign *= -1

        # find the unique and common spinorbitals
        so = np.block(list(map(lambda l: np.array(l), [
            [aligned[k] for k in range(len(aligned)) if aligned[k] not in dets[i]],
            [dets[i][k] for k in range(len(dets[j])) if dets[i][k] not in aligned],
            [aligned[k] for k in range(len(aligned)) if aligned[k] in dets[i]]
        ]))).astype(int)

        # apply the Slater-Condon rules and multiply by the sign
        if ((aligned - dets[i]) != 0).sum() == 0: Hci[i, j] = slater0(so) * sign
        if ((aligned - dets[i]) != 0).sum() == 1: Hci[i, j] = slater1(so) * sign
        if ((aligned - dets[i]) != 0).sum() == 2: Hci[i, j] = slater2(so) * sign

        # fill the lower triangle
        Hci[j, i] = Hci[i, j]

# solve the eigensystem and assign energy
eci, Cci = np.linalg.eigh(Hci); E_FCI = eci[0] - E_HF

# print the results
print("    FCI ENERGY: {:.8f}".format(E_HF + E_FCI + VNN))
\end{lstlisting}

\section*{\texorpdfstring{Coupled Cluster Singles and Doubles\label{sec:cc_code_solution}}{Coupled Cluster Singles and Doubles}}

\raggedbottom\begin{lstlisting}[language=Python, caption={\acrshort{ccd} and \acrshort{ccsd} method exercise code solution.}, label=code:cc_solution]
# energy containers for all the CC methods
E_CCD, E_CCD_P, E_CCSD, E_CCSD_P = 0, 1, 0, 1

# initialize the first guess for the t-amplitudes as zeros
t1, t2 = np.zeros((2 * nvirt, 2 * nocc)), np.zeros(2 * [2 * nvirt] + 2 * [2 * nocc])

# CCD loop
if args.ccd:
    while abs(E_CCD - E_CCD_P) > args.threshold:

        # collect all the distinct LCCD terms
        lccd1 = 0.5 * np.einsum("abcd,cdij->abij", Jmsa[v, v, v, v], t2, optimize=True)
        lccd2 = 0.5 * np.einsum("klij,abkl->abij", Jmsa[o, o, o, o], t2, optimize=True)
        lccd3 = 1.0 * np.einsum("akic,bcjk->abij", Jmsa[v, o, o, v], t2, optimize=True)

        # apply the permuation operator and add it to the corresponding LCCD terms
        lccd3 += lccd3.transpose(1, 0, 3, 2) - lccd3.swapaxes(0, 1) - lccd3.swapaxes(2, 3)

        # collect all the remaining CCD terms
        ccd1 = -0.50 * np.einsum("klcd,acij,bdkl->abij",
            Jmsa[o, o, v, v], t2, t2, optimize=True
        )
        ccd2 = -0.50 * np.einsum("klcd,abik,cdjl->abij",
            Jmsa[o, o, v, v], t2, t2, optimize=True
        )
        ccd3 = +0.25 * np.einsum("klcd,cdij,abkl->abij",
            Jmsa[o, o, v, v], t2, t2, optimize=True
        )
        ccd4 = +1.00 * np.einsum("klcd,acik,bdjl->abij",
            Jmsa[o, o, v, v], t2, t2, optimize=True
        )

        # permutation operators
        ccd1 -= ccd1.swapaxes(0, 1);
        ccd2 -= ccd2.swapaxes(2, 3);
        ccd4 -= ccd4.swapaxes(2, 3)

        # update the t-amplitudes
        t2 = (Jmsa[v, v, o, o] + lccd1 + lccd2 + lccd3 + ccd1 + ccd2 + ccd3 + ccd4) / Emsd

        # evaluate the energy
        E_CCD_P, E_CCD = E_CCD, 0.25 * np.einsum("ijab,abij", Jmsa[o, o, v, v], t2)

    # print the CCD energy
    print("    CCD ENERGY: {:.8f}".format(E_HF + E_CCD + VNN))

# CCSD loop
if args.ccsd:
    while abs(E_CCSD - E_CCSD_P) > args.threshold:

        # define taus
        tau, ttau = t2.copy(), t2.copy()

        # add contributions to the tilde tau
        ttau += 0.5 * np.einsum("ai,bj->abij", t1, t1, optimize=True).swapaxes(0, 0)
        ttau -= 0.5 * np.einsum("ai,bj->abij", t1, t1, optimize=True).swapaxes(2, 3)

        # add the contributions to tau
        tau += np.einsum("ai,bj->abij", t1, t1, optimize=True).swapaxes(0, 0)
        tau -= np.einsum("ai,bj->abij", t1, t1, optimize=True).swapaxes(2, 3)

        # define the deltas for Fae and Fmi
        dae, dmi = np.eye(2 * nvirt), np.eye(2 * nocc)

        # define Fae, Fmi and Fme
        Fae, Fmi, Fme = (1 - dae) * Fms[v, v], (1 - dmi) * Fms[o, o], Fms[o, v].copy()

        # add the contributions to Fae
        Fae -= 0.5 * np.einsum("me,am->ae",     Fms[o, v],        t1,   optimize=True)
        Fae += 1.0 * np.einsum("mafe,fm->ae",   Jmsa[o, v, v, v], t1,   optimize=True)
        Fae -= 0.5 * np.einsum("mnef,afmn->ae", Jmsa[o, o, v, v], ttau, optimize=True)

        # add the contributions to Fmi
        Fmi += 0.5 * np.einsum("me,ei->mi",     Fms[o, v],        t1,   optimize=True)
        Fmi += 1.0 * np.einsum("mnie,en->mi",   Jmsa[o, o, o, v], t1,   optimize=True)
        Fmi += 0.5 * np.einsum("mnef,efin->mi", Jmsa[o, o, v, v], ttau, optimize=True)

        # add the contributions to Fme
        Fme += np.einsum("mnef,fn->me", Jmsa[o, o, v, v], t1, optimize=True)

        # define Wmnij, Wabef and Wmbej
        Wmnij = Jmsa[o, o, o, o].copy()
        Wabef = Jmsa[v, v, v, v].copy()
        Wmbej = Jmsa[o, v, v, o].copy()

        # define some complementary variables used in the Wmbej intermediate
        t12  = 0.5 * t2 + np.einsum("fj,bn->fbjn", t1, t1,  optimize=True)

        # add contributions to Wmnij
        Wmnij += 0.25 * np.einsum("efij,mnef->mnij", tau, Jmsa[o, o, v, v], optimize=True)
        Wabef += 0.25 * np.einsum("abmn,mnef->abef", tau, Jmsa[o, o, v, v], optimize=True)
        Wmbej += 1.00 * np.einsum("fj,mbef->mbej",   t1,  Jmsa[o, v, v, v], optimize=True) 
        Wmbej -= 1.00 * np.einsum("bn,mnej->mbej",   t1,  Jmsa[o, o, v, o], optimize=True) 
        Wmbej -= 1.00 * np.einsum("fbjn,mnef->mbej", t12, Jmsa[o, o, v, v], optimize=True) 

        # define the permutation arguments for Wmnij and Wabef and add them
        PWmnij = np.einsum("ej,mnie->mnij", t1, Jmsa[o, o, o, v], optimize=True)
        PWabef = np.einsum("bm,amef->abef", t1, Jmsa[v, o, v, v], optimize=True)

        # add the permutations to Wmnij and Wabef
        Wmnij += PWmnij - PWmnij.swapaxes(2, 3)
        Wabef += PWabef.swapaxes(0, 1) - PWabef

        # define the right hand side of the T1 and T2 amplitude equations
        rhs_T1, rhs_T2 = Fms[v, o].copy(), Jmsa[v, v, o, o].copy()

        # calculate the right hand side of the CCSD equation for T1
        rhs_T1 += 1.0 * np.einsum("ei,ae->ai",     t1, Fae,              optimize=True)
        rhs_T1 -= 1.0 * np.einsum("am,mi->ai",     t1, Fmi,              optimize=True)
        rhs_T1 += 1.0 * np.einsum("aeim,me->ai",   t2, Fme,              optimize=True)
        rhs_T1 -= 1.0 * np.einsum("fn,naif->ai",   t1, Jmsa[o, v, o, v], optimize=True)
        rhs_T1 -= 0.5 * np.einsum("efim,maef->ai", t2, Jmsa[o, v, v, v], optimize=True)
        rhs_T1 -= 0.5 * np.einsum("aemn,nmei->ai", t2, Jmsa[o, o, v, o], optimize=True)

        # contracted F matrices that used in the T2 equations
        Faet = np.einsum("bm,me->be", t1, Fme, optimize=True)
        Fmet = np.einsum("ej,me->mj", t1, Fme, optimize=True)

        # define the permutation arguments for all terms in the equation for T2
        P1  = np.einsum("aeij,be->abij",    t2,     Fae - 0.5 * Faet, optimize=True)
        P2  = np.einsum("abim,mj->abij",    t2,     Fmi + 0.5 * Fmet, optimize=True)
        P3  = np.einsum("aeim,mbej->abij",  t2,     Wmbej,            optimize=True)
        P3 -= np.einsum("ei,am,mbej->abij", t1, t1, Jmsa[o, v, v, o], optimize=True)
        P4  = np.einsum("ei,abej->abij",    t1,     Jmsa[v, v, v, o], optimize=True)
        P5  = np.einsum("am,mbij->abij",    t1,     Jmsa[o, v, o, o], optimize=True)

        # calculate the right hand side of the CCSD equation for T2
        rhs_T2 += 0.5 * np.einsum("abmn,mnij->abij", tau, Wmnij, optimize=True)
        rhs_T2 += 0.5 * np.einsum("efij,abef->abij", tau, Wabef, optimize=True)
        rhs_T2 += P1.transpose(0, 1, 2, 3) - P1.transpose(1, 0, 2, 3)
        rhs_T2 -= P2.transpose(0, 1, 2, 3) - P2.transpose(0, 1, 3, 2)
        rhs_T2 += P3.transpose(0, 1, 2, 3) - P3.transpose(0, 1, 3, 2)
        rhs_T2 -= P3.transpose(1, 0, 2, 3) - P3.transpose(1, 0, 3, 2)
        rhs_T2 += P4.transpose(0, 1, 2, 3) - P4.transpose(0, 1, 3, 2)
        rhs_T2 -= P5.transpose(0, 1, 2, 3) - P5.transpose(1, 0, 2, 3)
        
        # Update T1 and T2 amplitudes and save the previous iteration
        t1, t2 = rhs_T1 / Emss, rhs_T2 / Emsd; E_CCSD_P = E_CCSD

        # evaluate the energy
        E_CCSD  = 1.00 * np.einsum("ia,ai",      Fms[o, v],        t1    )
        E_CCSD += 0.25 * np.einsum("ijab,abij",  Jmsa[o, o, v, v], t2    )
        E_CCSD += 0.50 * np.einsum("ijab,ai,bj", Jmsa[o, o, v, v], t1, t1)

    # print the CCSD energy
    print("   CCSD ENERGY: {:.8f}".format(E_HF + E_CCSD + VNN))
\end{lstlisting}


\section*{Quantum dynamics in real time}

\lstset{style=mystyle}
\lstinputlisting[caption=Solution: quantum dynamics in real-time,language=Python]{codes/solutions/rt_quantum_dynamics.py}

\section*{Energy spectrum and autocorrelation function}

\lstset{style=mystyle}
\lstinputlisting[caption=Solution: autocorrelation fucntion,language=Python]{codes/solutions/autocorr.py}

\section*{Imaginary-time dynamics and stationary states}

\lstset{style=mystyle}
\lstinputlisting[caption=Solution: quantum dynamics in imaginary-time,language=Python]{codes/solutions/it_quantum_dynamics.py}